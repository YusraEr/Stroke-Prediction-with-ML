\documentclass[10pt]{article}
\usepackage{times}
\usepackage{geometry}
\usepackage{multicol}
\usepackage{setspace}
\usepackage{parskip} 
\usepackage{booktabs} 
\usepackage{caption}
\usepackage{graphicx}
\usepackage{float}
\usepackage{fancyhdr}   
\usepackage{array}
\usepackage{amsmath} 
\usepackage{booktabs}
\usepackage{hhline}
\bibliographystyle{ieeetr}

% Configure table caption style
\captionsetup[table]{font=small,skip=3pt}
\renewcommand{\tablename}{Table}

% Configure figure caption style
\captionsetup[figure]{font=small,justification=centering,skip=3pt}
\renewcommand{\figurename}{Figure}

% Configure footer
\pagestyle{fancy}
\fancyhf{}
\fancyfoot[C]{\rule{\textwidth}{0.4pt}\\ \thepage} % Tambahkan nomor halaman

\makeatletter
\renewcommand\section{\@startsection{section}{1}{\z@}%
  {-3.5ex \@plus -1ex \@minus -.2ex}%
  {2.3ex \@plus.2ex}%
  {\normalfont\normalsize\bfseries}}
\renewcommand\subsection{\@startsection{subsection}{2}{\z@}%
  {-3.25ex\@plus -1ex \@minus -.2ex}%
  {1.5ex \@plus .2ex}%
  {\normalfont\itshape\normalsize}}
\makeatother

\geometry{
    a4paper,
    left=25mm,
    right=25mm,
    top=25mm,
    bottom=25mm
}

\setstretch{1.0} % Single spacing
\setlength{\parskip}{6pt} % Set spacing between paragraphs

\renewcommand{\thesection}{\arabic{section}.}
\renewcommand{\thesubsection}{\arabic{section}.\arabic{subsection}}

\begin{document}
{\fontsize{15}{18}\selectfont
\begin{center}
    Stroke Prediction using Machine Learning
\end{center}}

\begin{center}
    {\fontsize{10}{12}\selectfont Yusra Erlangga Putra$^1$, Sheryl Anastasya$^{2}$, Resky Auliyah Kartini Askin$^3$, Ivan Betrandi$^4$, Amaliah Diah$^5$}\\
    {\fontsize{9}{11}\selectfont
    \begin{tabular}{c}
        Departemen Matematika, Fakultas Matematika dan Ilmu Pengetahuan Alam, Universitas Hasanuddin, Makassar, Indonesia \\
        $^1$yusraerlangg@gmail.com, $^{2}$jane.doe@email.com                                                              \\
    \end{tabular}}
\end{center}

\vspace{0.5cm}

{\fontsize{10}{12}\selectfont\noindent\textbf{\textit{Abstrak}}}

\noindent{\fontsize{9}{11}\selectfont\itshape%
    Stroke merupakan penyebab utama kecacatan jangka panjang dan kematian di seluruh dunia, dengan risiko yang meningkat seiring dengan bertambahnya usia serta adanya faktor risiko seperti hipertensi dan diabetes. Tujuan penelitian ini adalah mengembangkan model prediksi dini yang akurat untuk stroke sehingga memungkinkan intervensi perawatan kesehatan preventif yang efektif. Penelitian ini menggunakan algoritma Random Forest dan Support Vector Machine (SVM). Karena ketidakseimbangan pada himpunan data, teknik Synthetic Minority Oversampling Technique (SMOTE) diterapkan untuk meningkatkan representasi data minoritas. Model dioptimalkan melalui penyesuaian hiperparameter menggunakan Optimasi Bayesian. Evaluasi model dilakukan dengan metrik akurasi, presisi, sensitivitas, dan skor-F1, dengan validasi silang untuk memastikan keandalan pada data yang belum terlihat. Hasil eksperimen menunjukkan bahwa algoritma SVM dengan SMOTE dan Optimasi Bayesian mencapai akurasi tertinggi. Temuan ini menunjukkan bahwa model pembelajaran mesin yang dioptimalkan dapat memberikan kontribusi signifikan dalam prediksi dini stroke dan mendukung pengambilan keputusan dalam sistem perawatan kesehatan preventif.
    \par}

\noindent{\fontsize{9}{11}\selectfont
    \textit{\textbf{Kata kunci:} Prediksi Stroke, pembelajaran mesin, Random Forest, Support Vector Machine, SMOTE, Bayesian Optimization}
    \par}

\begin{multicols}{2}
    \setlength{\columnsep}{0.4pt} %
    \raggedcolumns%
    \sloppy

    \section{Introduction}
    \subsection{Latar Belakang}
    Stroke merupakan masalah kesehatan global yang serius, menempati posisi kedua
    sebagai penyebab kematian tertinggi dan posisi ketiga sebagai penyebab
    kecacatan di dunia. Menurut data WHO, satu dari empat orang berisiko mengalami
    stroke dalam masa hidupnya, dengan 70\% kasus terjadi di negara-negara
    berpenghasilan rendah dan menengah\cite{who2024stroke}. Kompleksitas faktor
    risiko stroke yang meliputi tekanan darah tinggi, kolesterol, diabetes,
    obesitas, dan gaya hidup, menjadikan prediksi stroke sebagai tantangan yang
    signifikan dalam dunia kesehatan.\cite{emon2020performance}

    Prediksi dini stroke menggunakan metode konvensional seringkali terbatas dalam
    kemampuannya mengintegrasikan berbagai faktor risiko secara simultan.
    Perkembangan pembelajaran mesin membuka peluang baru dalam meningkatkan akurasi
    prediksi stroke. Dengan kemampuannya menganalisis data kompleks dalam jumlah
    besar, teknik pembelajaran mesin dapat mengidentifikasi pola-pola tersembunyi
    dan korelasi antar berbagai faktor risiko yang sulit dideteksi melalui metode
    konvensional.

    Penelitian ini mengusulkan pendekatan inovatif dengan mengombinasikan algoritma
    Random Forest dan Support Vector Machine (SVM) untuk prediksi stroke. Kedua
    algoritma ini dipilih karena kemampuannya dalam menangani data kesehatan yang
    kompleks dan menghasilkan model prediksi yang akurat. Untuk mengatasi masalah
    ketidakseimbangan data yang umum dalam kasus medis, diterapkan teknik SMOTE
    (Synthetic Minority Oversampling Technique). Selanjutnya, optimasi Bayesian
    digunakan untuk meningkatkan performa model melalui penyetelan hiperparameter
    yang optimal. Pendekatan komprehensif ini diharapkan dapat menghasilkan sistem
    prediksi stroke yang lebih andal dan aplikatif dalam praktik klinis.

    \subsection{Literature Review}
    % Brief review of related research goes here...

    \subsection{Research Rationale}
    % Reason for conducting this research goes here...

    \subsection{Research Questions and Objectives}
    Penelitian ini didasari oleh beberapa pertanyaan penelitian yang berkaitan
    dengan keefektifan algoritma Random Forest dan Support Vector Machine dalam
    memprediksi risiko stroke berdasarkan berbagai faktor kesehatan. Selain itu,
    penelitian ini juga mengkaji sejauh mana penerapan teknik SMOTE dapat
    meningkatkan kinerja model dalam menangani ketidakseimbangan data pada kasus
    prediksi stroke, serta bagaimana Optimasi Bayesian dapat memengaruhi kinerja
    model dalam hal akurasi dan presisi prediksi stroke.

    Berdasarkan pertanyaan penelitian tersebut, penelitian ini bertujuan untuk
    mengembangkan dan membandingkan model prediksi stroke menggunakan algoritma
    Random Forest dan Support Vector Machine untuk mengidentifikasi pendekatan yang
    paling efektif. Penelitian ini juga akan mengevaluasi dampak penerapan SMOTE
    dalam meningkatkan kualitas prediksi pada kasus dengan sebaran data yang tidak
    seimbang, serta mengoptimalkan parameter model menggunakan Optimasi Bayesian
    untuk mencapai kinerja prediksi yang optimal. Tujuan akhir penelitian ini
    adalah menghasilkan model prediksi stroke yang dapat diterapkan dalam sistem
    pendukung keputusan klinis untuk deteksi dini risiko stroke.

    \section{Research Methods}
    \subsection{Dataset Description}
    Dataset yang digunakan dalam penelitian ini berasal dari \textit{Kaggle Stroke
        Prediction Dataset}\cite{kaggleStrokePredictionDataset}, yang berisi informasi
    kesehatan dari 5110 pasien. Dataset ini mencakup berbagai parameter demografis
    dan faktor risiko kesehatan yang berpotensi terkait dengan kejadian stroke.
    Data dikumpulkan dari berbagai fasilitas kesehatan dan mencakup informasi
    seperti usia, jenis kelamin, berbagai penyakit, dan gaya hidup pasien.

    \begin{table}[H]
        \centering
        {\fontsize{8}{10}\selectfont
            \caption{Deskripsi Fitur Dataset Stroke Prediction}
            \begin{tabular}{>{\raggedright\arraybackslash}p{2cm}>{\raggedright\arraybackslash}p{3cm}>{\raggedright\arraybackslash}p{1.5cm}}
                \specialrule{0.07em}{0em}{0.06em}
                \specialrule{0.07em}{0em}{0.4em}
                \textbf{Fitur}      & \textbf{Deskripsi}                                                                                                       & \textbf{Tipe Data} \\
                \midrule
                id                  & Identifier unik setiap pasien                                                                                            & Numerik            \\
                gender              & Jenis kelamin pasien (\textit{Male}, \textit{Female}, \textit{Other})                                                    & Kategorikal        \\
                age                 & Usia pasien                                                                                                              & Numerik            \\
                hypertension        & 0 jika pasien tidak memiliki hipertensi, 1 jika memiliki hipertensi                                                      & Biner              \\
                heart\_disease      & 0 jika pasien tidak memiliki penyakit jantung, 1 jika memiliki penyakit jantung                                          & Biner              \\
                ever\_married       & Status pernikahan (\textit{Yes}, \textit{No})                                                                            & Kategorikal        \\
                work\_type          & Tipe pekerjaan (\textit{children}, \textit{Govt\_job}, \textit{Never\_worked}, \textit{Private}, \textit{Self-employed}) & Kategorikal        \\
                Residence\_type     & Tipe tempat tinggal (\textit{Rural}, \textit{Urban})                                                                     & Kategorikal        \\
                avg\_glucose\_level & Level glukosa rata-rata dalam darah (\textit{Average Glucose Level})                                                     & Numerik            \\
                bmi                 & Body Mass Index                                                                                                          & Numerik            \\
                smoking\_status     & Status merokok (\textit{formerly smoked}, \textit{never smoked}, \textit{smokes}, \textit{Unknown})                      & Kategorikal        \\
                stroke              & 1 jika pasien pernah stroke, 0 jika tidak                                                                                & Biner              \\
                \bottomrule
            \end{tabular}}
    \end{table}

    \subsection{Performance Metrics}
    Dalam evaluasi model klasifikasi stroke, kami menggunakan beberapa metrik
    performa standar yang diperoleh dari \textit{confusion matrix}.

    \textbf{Accuracy} mengukur proporsi total prediksi yang benar dibandingkan dengan semua kasus.
    \begin{equation}
        Accuracy = \frac{TP + TN}{TP + TN + FP + FN}
    \end{equation}

    \textbf{Precision} mengukur proporsi prediksi positif yang benar-benar positif.
    \begin{equation}
        Precision = \frac{TP}{TP + FP}
    \end{equation}

    \textbf{Recall (Sensitivity)} mengukur proporsi kasus positif yang berhasil diidentifikasi.
    \begin{equation}
        Recall = \frac{TP}{TP + FN}
    \end{equation}

    \textbf{F1-Score} merupakan rata-rata harmonik dari \textit{precision} dan \textit{recall}.
    \begin{equation}
        \text{F1-Score} = \frac{2 \times \text{Precision} \times \text{Recall}}{\text{Precision} + \text{Recall}}
    \end{equation}

    Dalam konteks ini, TP (\textit{True Positive}) adalah jumlah kasus stroke yang
    diprediksi benar sebagai stroke. TN (\textit{True Negative}) merupakan jumlah
    kasus non-stroke yang diprediksi benar sebagai non-stroke. FP (\textit{False
        Positive}) menunjukkan jumlah kasus non-stroke yang salah diprediksi sebagai
    stroke. FN (\textit{False Negative}) adalah jumlah kasus stroke yang salah
    diprediksi sebagai non-stroke.

    Pemilihan metrik-metrik ini didasarkan pada karakteristik dataset yang tidak
    seimbang, sehingga F1-score menjadi sangat penting karena memberikan gambaran
    yang lebih baik tentang performa model pada kasus yang tidak seimbang
    dibandingkan dengan \textit{accuracy} saja.

    \subsection{Formulas}
    % Example formula
    \begin{equation}
        E = mc^2
        \label{eq:1}
    \end{equation}

    \section{Results and Discussion}
    % \end{multicols}

    % Example of figure spanning both columns
    % \begin{figure*}[!htb]
    %     \centering
    %     \includegraphics[width=0.8\textwidth]{example-image}
    %     \caption{Description of your figure. Make sure to reference it in text using Fig.~\ref{fig:example}.}
    %     \label{fig:example}
    % \end{figure*}

    % Example of single-column figure
    \begin{figure}[H]
        \centering
        \includegraphics[width=\columnwidth]{example-image}
        \caption{A single-column figure example.}%
        \label{fig:single-col}
    \end{figure}

    % Use table* to span both columns
    \begin{table}[H]
        \centering
        {\fontsize{8}{10}\selectfont
            \caption{Dataset characteristics for stroke prediction}
            \begin{tabular}{lcc}
                \toprule
                Characteristic   & Value & Percentage \\
                \midrule
                Total Samples    & 5110  & 100\%      \\
                Stroke Cases     & 249   & 4.87\%     \\
                Non-stroke Cases & 4861  & 95.13\%    \\
                \bottomrule
            \end{tabular}}
    \end{table}

    % \begin{multicols}{2}
    % Results and discussion content goes here...

    \section{Conclusion}
    % Conclusion content goes here...

    \section*{Acknowledgements}
    % Acknowledgements content goes here...

    \bibliography{references}

\end{multicols}
\end{document}
