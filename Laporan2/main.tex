\documentclass[10pt]{article}
\usepackage{times}
\usepackage{geometry}
\usepackage{multicol}
\usepackage{setspace}
\usepackage{parskip} % Add package for paragraph spacing
\usepackage{booktabs}  % For professional tables
\usepackage{caption}   % For caption formatting
\usepackage{graphicx}    % For including images
\usepackage{float}       % For better figure placement
\usepackage{fancyhdr}    % For custom headers and footers
\usepackage{array}

% Configure table caption style
\captionsetup[table]{font=small,skip=3pt}
\renewcommand{\tablename}{Table}

% Configure figure caption style
\captionsetup[figure]{font=small,justification=centering,skip=3pt}
\renewcommand{\figurename}{Figure}

% Configure footer
\pagestyle{fancy}
\fancyhf{}
\fancyfoot[C]{\rule{\textwidth}{0.4pt}} % Add long line to footer

\makeatletter
\renewcommand\section{\@startsection{section}{1}{\z@}%
  {-3.5ex \@plus -1ex \@minus -.2ex}%
  {2.3ex \@plus.2ex}%
  {\normalfont\normalsize\bfseries}}
\renewcommand\subsection{\@startsection{subsection}{2}{\z@}%
  {-3.25ex\@plus -1ex \@minus -.2ex}%
  {1.5ex \@plus .2ex}%
  {\normalfont\itshape\normalsize}}
\makeatother

\geometry{
    a4paper,
    left=25mm,
    right=25mm,
    top=25mm,
    bottom=25mm
}

\setstretch{1.0} % Single spacing
\setlength{\parskip}{6pt} % Set spacing between paragraphs

\renewcommand{\thesection}{\arabic{section}.}
\renewcommand{\thesubsection}{\arabic{section}.\arabic{subsection}}

\begin{document}
{\fontsize{15}{18}\selectfont
\begin{center}
    Stroke Prediction using Machine Learning
\end{center}}

\begin{center}
    {\fontsize{10}{12}\selectfont Yusra Erlangga Putra$^1$, Sheryl Anastasya$^{2}$, Resky Auliyah Kartini Askin$^3$, Ivan Betrandi$^4$, Amaliah Diah$^5$}\\
    {\fontsize{9}{11}\selectfont
    \begin{tabular}{c}
        Departemen Matematika, Fakultas Matematika dan Ilmu Pengetahuan Alam, Universitas Hasanuddin, Makassar, Indonesia \\
        $^1$yusraerlangg@gmail.com, $^{2}$jane.doe@email.com                                                              \\
        $^*$Corresponding author
    \end{tabular}}
\end{center}

{\fontsize{10}{12}\selectfont\noindent\textbf{\textit{Abstrak}}}
\vspace{0.3em}

\noindent{\fontsize{9}{11}\selectfont\itshape
    Stroke merupakan penyebab utama kecacatan jangka panjang dan kematian di seluruh dunia, dengan risiko yang meningkat seiring bertambahnya usia serta adanya faktor risiko seperti hipertensi dan diabetes. Tujuan penelitian ini adalah mengembangkan model prediksi dini dan akurat untuk stroke guna memungkinkan intervensi perawatan kesehatan preventif yang efektif. Studi ini menggunakan algoritma Random Forest dan Support Vector Machine (SVM). Karena ketidakseimbangan pada dataset, teknik Synthetic Minority Oversampling Technique (SMOTE) diterapkan untuk meningkatkan representasi data minoritas. Model dioptimalkan melalui penyetelan hiperparameter menggunakan Bayesian Optimization. Evaluasi model dilakukan dengan metrik akurasi, precision, recall, dan F1-score, dengan validasi silang untuk memastikan keandalan pada data yang belum terlihat. Hasil eksperimen menunjukkan bahwa algoritma SVM dengan SMOTE dan optimasi Bayesian mencapai akurasi tertinggi. Temuan ini menunjukkan bahwa model pembelajaran mesin yang dioptimalkan dapat memberikan kontribusi signifikan dalam prediksi dini stroke dan mendukung pengambilan keputusan dalam sistem perawatan kesehatan preventif.
    \par}

\noindent{\fontsize{9}{11}\selectfont
    \textit{\textbf{Kata kunci:} Prediksi Stroke, pembelajaran mesin, Random Forest, Support Vector Machine, SMOTE, Bayesian Optimization}
    \par}

\begin{multicols}{2}
    \setlength{\columnsep}{0.4pt} %
    \raggedcolumns%
    \sloppy % Justify text alignment

    \section{Introduction}
    \subsection{Backgorund}
    % Background content goes here...

    \subsection{Literature Review}
    % Brief review of related research goes here...

    \subsection{Research Rationale}
    % Reason for conducting this research goes here...

    \subsection{Research Questions and Objectives}
    % Research questions and objectives go here...

    \section{Research Methods}
    % Research methods content goes here...

    \subsection{Formulas}
    % Example formula
    \begin{equation}
        E = mc^2
        \label{eq:1}
    \end{equation}

    \section{Results and Discussion}
    % \end{multicols}

    % Example of figure spanning both columns
    \begin{figure*}[!htb]
        \centering
        \includegraphics[width=0.8\textwidth]{example-image}
        \caption{Description of your figure. Make sure to reference it in text using Fig.~\ref{fig:example}.}
        \label{fig:example}
    \end{figure*}

    % Example of single-column figure
    \begin{figure}[H]
        \centering
        \includegraphics[width=\columnwidth]{example-image}
        \caption{A single-column figure example.}
        \label{fig:single-col}
    \end{figure}

    % Use table* to span both columns
    \begin{table}[H]
        \centering
        {\fontsize{8}{10}\selectfont
            \caption{Dataset characteristics for stroke prediction}
            \begin{tabular}{lcc}
                \toprule
                Characteristic   & Value & Percentage \\
                \midrule
                Total Samples    & 5110  & 100\%      \\
                Stroke Cases     & 249   & 4.87\%     \\
                Non-stroke Cases & 4861  & 95.13\%    \\
                \bottomrule
            \end{tabular}}
    \end{table}

    % \begin{multicols}{2}
    % Results and discussion content goes here...

    \section{Conclusion}
    % Conclusion content goes here...

    \section*{Acknowledgements}
    % Acknowledgements content goes here...

    \begin{thebibliography}{9}
        \bibitem{ref1} % Reference 1
        \bibitem{ref2} % Reference 2
        \bibitem{ref3} % Reference 3
        % Add more references as needed
    \end{thebibliography}

\end{multicols}
\end{document}
