\documentclass[10pt]{article}
\usepackage{times}
\usepackage{geometry}
\usepackage{multicol}
\usepackage{setspace}
\usepackage{parskip} % Add package for paragraph spacing
\usepackage{booktabs}  % For professional tables
\usepackage{caption}   % For caption formatting
\usepackage{graphicx}    % For including images
\usepackage{float}       % For better figure placement
\usepackage{fancyhdr}    % For custom headers and footers
\usepackage{array}
\bibliographystyle{ieeetr}

% Configure table caption style
\captionsetup[table]{font=small,skip=3pt}
\renewcommand{\tablename}{Table}

% Configure figure caption style
\captionsetup[figure]{font=small,justification=centering,skip=3pt}
\renewcommand{\figurename}{Figure}

% Configure footer
\pagestyle{fancy}
\fancyhf{}
\fancyfoot[C]{\rule{\textwidth}{0.4pt}} % Add long line to footer

\makeatletter
\renewcommand\section{\@startsection{section}{1}{\z@}%
  {-3.5ex \@plus -1ex \@minus -.2ex}%
  {2.3ex \@plus.2ex}%
  {\normalfont\normalsize\bfseries}}
\renewcommand\subsection{\@startsection{subsection}{2}{\z@}%
  {-3.25ex\@plus -1ex \@minus -.2ex}%
  {1.5ex \@plus .2ex}%
  {\normalfont\itshape\normalsize}}
\makeatother

\geometry{
    a4paper,
    left=25mm,
    right=25mm,
    top=25mm,
    bottom=25mm
}

\setstretch{1.0} % Single spacing
\setlength{\parskip}{6pt} % Set spacing between paragraphs

\renewcommand{\thesection}{\arabic{section}.}
\renewcommand{\thesubsection}{\arabic{section}.\arabic{subsection}}

\begin{document}
{\fontsize{15}{18}\selectfont
\begin{center}
    Stroke Prediction using Machine Learning
\end{center}}

\begin{center}
    {\fontsize{10}{12}\selectfont Yusra Erlangga Putra$^1$, Sheryl Anastasya$^{2}$, Resky Auliyah Kartini Askin$^3$, Ivan Betrandi$^4$, Amaliah Diah$^5$}\\
    {\fontsize{9}{11}\selectfont
    \begin{tabular}{c}
        Departemen Matematika, Fakultas Matematika dan Ilmu Pengetahuan Alam, Universitas Hasanuddin, Makassar, Indonesia \\
        $^1$yusraerlangg@gmail.com, $^{2}$jane.doe@email.com                                                              \\
    \end{tabular}}
\end{center}

\vspace{0.5cm}

{\fontsize{10}{12}\selectfont\noindent\textbf{\textit{Abstrak}}}

\noindent{\fontsize{9}{11}\selectfont\itshape%
    Stroke merupakan penyebab utama kecacatan jangka panjang dan kematian di seluruh dunia, dengan risiko yang meningkat seiring bertambahnya usia serta adanya faktor risiko seperti hipertensi dan diabetes. Tujuan penelitian ini adalah mengembangkan model prediksi dini dan akurat untuk stroke guna memungkinkan intervensi perawatan kesehatan preventif yang efektif. Studi ini menggunakan algoritma Random Forest dan Support Vector Machine (SVM). Karena ketidakseimbangan pada dataset, teknik Synthetic Minority Oversampling Technique (SMOTE) diterapkan untuk meningkatkan representasi data minoritas. Model dioptimalkan melalui penyetelan hiperparameter menggunakan Bayesian Optimization. Evaluasi model dilakukan dengan metrik akurasi, precision, recall, dan F1-score, dengan validasi silang untuk memastikan keandalan pada data yang belum terlihat. Hasil eksperimen menunjukkan bahwa algoritma SVM dengan SMOTE dan optimasi Bayesian mencapai akurasi tertinggi. Temuan ini menunjukkan bahwa model pembelajaran mesin yang dioptimalkan dapat memberikan kontribusi signifikan dalam prediksi dini stroke dan mendukung pengambilan keputusan dalam sistem perawatan kesehatan preventif.
    \par}

\noindent{\fontsize{9}{11}\selectfont
    \textit{\textbf{Kata kunci:} Prediksi Stroke, pembelajaran mesin, Random Forest, Support Vector Machine, SMOTE, Bayesian Optimization}
    \par}

\begin{multicols}{2}
    \setlength{\columnsep}{0.4pt} %
    \raggedcolumns%
    \sloppy % Justify text alignment

    \section{Introduction}
    \subsection{Latar Belakang}
    Stroke merupakan masalah kesehatan global yang serius, menempati posisi kedua
    sebagai penyebab kematian tertinggi dan posisi ketiga sebagai penyebab
    kecacatan di dunia. Menurut data WHO, satu dari empat orang berisiko mengalami
    stroke dalam masa hidupnya, dengan 70\% kasus terjadi di negara-negara
    berpenghasilan rendah dan menengah\cite{who2024stroke}. Kompleksitas faktor risiko
    stroke yang meliputi tekanan darah tinggi, kolesterol, diabetes, obesitas, dan
    gaya hidup, menjadikan prediksi stroke sebagai tantangan yang signifikan dalam
    dunia kesehatan.\cite{emon2020performance}

    Prediksi dini stroke menggunakan metode konvensional seringkali terbatas dalam
    kemampuannya mengintegrasikan berbagai faktor risiko secara simultan.
    Perkembangan pembelajaran mesin membuka peluang baru dalam meningkatkan akurasi
    prediksi stroke. Dengan kemampuannya menganalisis data kompleks dalam jumlah
    besar, teknik pembelajaran mesin dapat mengidentifikasi pola-pola tersembunyi
    dan korelasi antar berbagai faktor risiko yang sulit dideteksi melalui metode
    konvensional.

    Penelitian ini mengusulkan pendekatan inovatif dengan mengombinasikan algoritma
    Random Forest dan Support Vector Machine (SVM) untuk prediksi stroke. Kedua
    algoritma ini dipilih karena kemampuannya dalam menangani data kesehatan yang
    kompleks dan menghasilkan model prediksi yang akurat. Untuk mengatasi masalah
    ketidakseimbangan data yang umum dalam kasus medis, diterapkan teknik SMOTE
    (Synthetic Minority Oversampling Technique). Selanjutnya, optimasi Bayesian
    digunakan untuk meningkatkan performa model melalui penyetelan hiperparameter
    yang optimal. Pendekatan komprehensif ini diharapkan dapat menghasilkan sistem
    prediksi stroke yang lebih andal dan aplikatif dalam praktik klinis.

    \subsection{Literature Review}
    % Brief review of related research goes here...

    \subsection{Research Rationale}
    % Reason for conducting this research goes here...

    \subsection{Research Questions and Objectives}
    % Research questions and objectives go here...

    \section{Research Methods}
    % Research methods content goes here...

    \subsection{Formulas}
    % Example formula
    \begin{equation}
        E = mc^2
        \label{eq:1}
    \end{equation}

    \section{Results and Discussion}
    % \end{multicols}

    % Example of figure spanning both columns
    % \begin{figure*}[!htb]
    %     \centering
    %     \includegraphics[width=0.8\textwidth]{example-image}
    %     \caption{Description of your figure. Make sure to reference it in text using Fig.~\ref{fig:example}.}
    %     \label{fig:example}
    % \end{figure*}

    % Example of single-column figure
    \begin{figure}[H]
        \centering
        \includegraphics[width=\columnwidth]{example-image}
        \caption{A single-column figure example.}%
        \label{fig:single-col}
    \end{figure}

    % Use table* to span both columns
    \begin{table}[H]
        \centering
        {\fontsize{8}{10}\selectfont
            \caption{Dataset characteristics for stroke prediction}
            \begin{tabular}{lcc}
                \toprule
                Characteristic   & Value & Percentage \\
                \midrule
                Total Samples    & 5110  & 100\%      \\
                Stroke Cases     & 249   & 4.87\%     \\
                Non-stroke Cases & 4861  & 95.13\%    \\
                \bottomrule
            \end{tabular}}
    \end{table}

    % \begin{multicols}{2}
    % Results and discussion content goes here...

    \section{Conclusion}
    % Conclusion content goes here...

    \section*{Acknowledgements}
    % Acknowledgements content goes here...

    \bibliography{references}

\end{multicols}
\end{document}
